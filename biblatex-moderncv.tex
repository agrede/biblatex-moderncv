%% biblatex-moderncv.tex
%% Copyright (c) 2013--2014 Alex J. Grede <agrede@gmail.com>
%%
%% This work may be distributed and/or modified under the
%% conditions of the LaTeX Project Public License, either version 1.3
%% of this license or (at your option) any later version.
%% The latest version of this license is in
%% http://www.latex-project.org/lppl.txt
%% and version 1.3 or later is part of all distributions of LaTeX
%% version 2005/12/01 or later.
%%
%% This work has the LPPL maintenance status `maintained'.
%%
%% The Current Maintainer of this work is Alex J. Grede
%%
%% This work consists of all files listed in manifest.txt.
\documentclass[a4paper]{ltxdoc}
\usepackage[utf8]{inputenc}
\usepackage[T1]{fontenc}
\usepackage{microtype}
\usepackage{minted}
\usepackage{listings}
\usepackage[english]{babel}
\usepackage{csquotes, lmodern}
\usepackage[
   breaklinks=true,
   colorlinks=true,
   linkcolor=blue,
   citecolor=blue,
   urlcolor=blue]{hyperref}
\usepackage[backend=biber, style=phys]{biblatex}

\addbibresource{biblatex-moderncv.bib}

\author{Alex J. Grede\thanks{E-mail:
    \href{mailto:agrede@gmail.com}{\texttt{agrede@gmail.com}}}}
\title{\pkg{biblatex-moderncv} -- A \pkg{biblatex} implementation for
  use with the \pkg{moderncv} document class%
  \footnote{This file describes v0.1.0, last revised 2014-04-11.}}
\date{Released 2014-04-11}

\providecommand*{\opt}[1]{\texttt{#1}}
\providecommand*{\pkg}[1]{\textsf{#1}}
\providecommand*{\bmcv}{\pkg{biblatex-moderncv}}

\let\DescribeOption\DescribeEnv

\RecordChanges

\begin{document}
\lstset{language=LaTeX}

\maketitle

This package is intended to provide a \citetitle{lehman_biblatex_2013}
style for use with \citetitle{danaux_moderncv_2013}. The style is
numeric, but has not been fully developed for citing within document
aside from publication listings. The style is a mix of the standard
style, borrows from styles of \textsc{AIP} and \textsc{AIP} (using
\citetitle{wright_biblatex-phys_2013} as a reference), \textsc{Nature}
(using \nocite{wright_biblatex-nature_2013} as a reference), and
personal my own preferences.

\section{Requirements}
\label{sec:requirements}

\begin{enumerate}
\item \citetitle{kime_biber_2013} must be used as the bibliography
  processor
\item \citetitle{kern_xcolor_2007} is required for
  \lstinline|\mkbibcolor|
\item The date formatting requires:
  \begin{enumerate}
  \item \citetitle{braams_babel_2014} for \LaTeX users
  \item \citetitle{charette_polyglossia_2013} for \XeTeX or \LuaTeX users
  \end{enumerate}
\item \citetitle{lehman_csquotes_2011} for article titles
\item \citetitle{arseneau_url_2013} is used to format hyperlinks
\end{enumerate}

\section{Style options}
\label{sec:style-options}

Additional style options are provided for use in CV publication
listings. Some default style options are changed as well.

\DescribeOption{doi}
\DescribeOption{eprint}
\DescribeOption{isbn}
\DescribeOption{url}
These options are turned off by default in \bmcv{}. Instead the
\opt{doi} and \opt{url} are used as they are in
\citetitle{wright_biblatex-phys_2013} and used to create links for the
title, journal title, proceedings title, or event title depending on
context.

\DescribeOption{articletitle}
Inherited from \citetitle{wright_biblatex-phys_2013}, this boolean
option will suppress the \opt{title} of \texttt{article} and
\texttt{inproceedings} entries when set to \lstinline|false|. The
default value is \lstinline|true|.

\DescribeOption{biblabel}
The default is set to \lstinline|superscript| which is used for
citations. \emph{This feature is not currently available}.

\DescribeOption{chaptertitle}
Similar to \opt{articletitle}, \opt{chaptertitle} is a boolean value
(default \lstinline|true|) which enables or disables the chapter title
in \texttt{incollection} entries.

\DescribeOption{pageranges}
This boolean value (default \lstinline|true|) determines if the full
range of page numbers is displayed or simply the first page. \emph{This
  feature is not currently available}.

\DescribeOption{maxnames}
\DescribeOption{maxcitenames}
The options \opt{maxnames} and \opt{maxcitenames} are set to \lstinline|999| and \lstinline|2|
respectively by default for \bmcv{}.

\DescribeOption{firstinits}
This defaults to \lstinline|true| for \bmcv{}

\DescribeOption{sorting}
The default value (\lstinline|ddnt|) is one that has been created for
\bmcv{}. This sorts entries by \emph{date} (descending),
then \texttt{name}, and finally \texttt{title}.

\DescribeOption{authorrole}
This boolean (default \listinline|false|) allows the use of the
\texttt{usera} field to detail what was contributed to the entry. This
functions similar to \texttt{addendum}, but is moved to
\texttt{usera} to prevent contamination when using the same bib file
or reference database for other work.

\DescribeOption{showevent}
This boolean (default \lstinline|false|) will display the
\texttt{eventtitle} and \texttt{eventdate} instead of the title of the
proceedings (\texttt{booktitle}) and publication date
(\texttt{date}). \emph{At present this does not change the sorting
  scheme}.

\DescribeOption{indauthorhash}
This boolean (default \lstinline|true|) prints the author name (with
hash set by \opt{authorhash}) using \lstinline|\indauthorhash|. By
default this uses \lstinline|\mkbibbold| and is explained in greater
detail later.

\DescribeOption{authorhash}
This is the hash value found in the \texttt{bbl} file generated by
\citetitle{kime_biber_2013} for the author to highlight. This requires
that the authors name be identical in each entry.

\DescribeOption{punctfont}
The default for \bmcv{} is \lstinline|true| more is explained in the
manual for \citetitle{lehman_biblatex_2013}.

\section{Author name formatting}
\label{sec:auth-name-form}

The commands for printing the authors name
\lstinline|\mkbibnamefirst|, \lstinline|\mkbibnamelast|,
\lstinline|\mkbibnameprefix|, and \lstinline|\mkbibnameaffix| are
changed from simply printing the name given to passing to a command
\lstinline|\bibnametest|. If both the authors name hash (from the
\texttt{bbl} file) and \opt{indauthorhash} is \lstinline|true|, the
name part is passed to \listinline|\indauthorhash| which by default
passes to \lstinline|\mkbibbold|. There are several ways to change
this behavior in the preamble for different effect. Some examples are
listed here.

\begin{minted}{latex}
% Bold name (default)
\renewcommand*{\indauthorhash}[1]{\mkbibbold{#1}}

% Color name <color>
\renewcommand*{\indauthorhash}[1]{\mkbibcolor{<color>}{#1}}

% Emphasize name
\renewcommand*{\indauthorhash}[1]{\mkbibemph{#1}}
\end{minted}

This method was inspired by a question on the TeX
StackExchange\cite{audrey_make_2012}.

The alias can be changed as normal (default is
\lstinline|last-first|). Additional options can be found in the
\citetitle{lehman_biblatex_2013} manual.

\section{Date format}
\label{sec:date-format}

The itemized bibliography has the first occurrence of the year printed
in the left of the list using a method from TeX
StackExchange\cite{audrey_sorted_2013}. In order to remove the year
from the entry date and print the month and day, the language file had
to be modified similar to using
\citetitle{kime_biblatex-apa_2013}. For \LaTeX users using babel the
following must be in the preamble:

\begin{minted}{latex}
\usepackage[american]{babel}
\usepackage{csquotes}
\usepackage[backend=biber, style=moderncv]{biblatex}
\DeclareLanguageMapping{american}{american-moderncv}
\end{minted}
or for \XeTeX and \LuaTeX:
\begin{minted}{latex}
\usepackage{polyglossia}
\setdefaultlanguage[variant=american]{english}
\usepackage{csquotes}
\usepackage[backend=biber, style=moderncv]{biblatex}
\DeclareLanguageMapping{english}{english-moderncv}
\end{minted}

Currently only the american style long date format is supported. The
year will appear only for cases where a range is used in \texttt{date}
where the start and end year differ. Also, in cases where the
\texttt{eventdate} is used and does not span within the same year as
\texttt{date}.

\section{Errors and omissions}
\label{sec:errors-omissions}

Improvement suggestions and bug reports should be submitted to
\url{https://github.com/agrede/biblatex-moderncv}, or e-mailed to
\href{mailto:agrede@gmail.com}{\texttt{agrede@gmail.com}}. Note this
is still under development.

\printbibliography

\end{document}


%%% Local Variables:
%%% mode: latex
%%% TeX-master: t
%%% zotero-collection: #("13" 0 2 (name "LaTeX"))
%%% End:
